\begin{spacing}{2}
    \section{总结和展望}
\end{spacing}

\subsection{全文总结}
遥感图像智能化处理是提取遥感图像的重要环节。现在正是我国遥感技术发展的黄金时期。遥感技术在军事国防和大众生活中均有广泛应用。因此大力发展遥感技术是十分有必要的。自动化提取遥感图像能够更高效地挖掘遥感图像中的数据。

此次论文希望可以使用卷积神经网络自动分割遥感图像。将分割医疗图像的U-Net网络移植到遥感图像上。由于此次数据集用于分割遥感图像中的建筑物。建筑物在城市与郊区的分布比例相差较大,因此提出了一种自动计算类别权重的带权交叉熵——类别平衡交叉熵。该方法在Inria Aerial Image Labeling Dataset数据集上表现较交叉熵函数好,使F1 Score提高了8.5\%。建筑识别率更高,分割结果更加完整。


\subsection{未来展望}

由于时间限制,很多设想未能实现,未来还能在以下几个方面进行改进:
\begin{itemize}
    \item 此次试验中未采用Batch Normalization策略\cite{ioffe2015batch}对网络进行优化,未来应该在网络中增加Batch Normalization层,再与未增加Batch Normalization层的网络比较实验结果;
    \item 这两种方法在面对大型建筑时都存在不能完整分割的问题。下一步应该针对不同大小的建筑做优化。如增大卷积核以增大感受野等;
    \item 由于图像进行了裁剪,因此在拼接处仍然可以看见拼接痕迹。下一步应该针对拼接痕迹做优化;
    \item 现在在针对输出图像做优化的方法还是简单的开运算,效果不是很好。在论文\cite{chandra2016fast}中提出使用条件随机场对输出图像做优化,未来应该尝试使用这种方法对输出的图像做出优化。
\end{itemize}